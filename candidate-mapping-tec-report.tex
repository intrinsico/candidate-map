%\documentclass[10pt,a4paper]{article}
\documentclass[10pt,a4paper]{llncs}

\usepackage{makeidx}  % allows for indexgeneration
\usepackage[utf8]{inputenc}
\usepackage[cmex10]{amsmath}
\usepackage{amsfonts}
\usepackage{hyperref}
\usepackage{color}
\usepackage{cleveref}
\usepackage{graphicx}
\usepackage{caption}
\usepackage{subcaption}

\newcommand{\todo}[1]{{\color{red}\textsf{\textbf{TODO}}: #1}}
\newcommand{\uri}[1]{\texttt{#1}}
\newcommand{\word}[1]{`#1'}
\newcommand{\model}[1]{\textsf{#1}}
\crefname{table}{Table}{Tables}

\title{Mapeamento de Candidatos}
\author{Alexandre Cançado Cardoso}
\institute{Intrinsic Soluções em Informática M.E.}
\date{\today}

\begin{document}
\maketitle

% \begin{abstract}
% \end{abstract}

\section{Introdução} \label{introducao}
\indent\indent Em um processo de anotação semântica a definição dos possíveis recursos candidatos para cada nome de entidade é uma tarefa fundamental. Pois é a partir destas relação que as anotações paras as entidades interessantes serão escolhidas. O conjunto das relações nome de entidades, recursos é denominado de mapa de candidatos. E pode ser definido de diversas formas.

Podendo o uso das diferentes formas de obtenção dos candidatos gerar mapas de qualidades distintas. Neste contexto, a avaliação de quais são as melhores estratégias de extração de candidatos se torna importante. Nas seções a seguir, quatro destas técnicas serão comparadas.


\section{Estratégias de Extração de Mapa de Candidatos} \label{extracao-mapa-candidatos}
\indent\indent As estratégias de extração de mapa de candidatos abordadas por este trabalho são a partir dos títulos, dos redirecionamentos, das desambiguações e das ocorrências no conteúdo das páginas.

O projeto DBpedia disponibilizam os arquivos com os títulos destas, redirecionamentos que levam para cada uma, e desambiguações (chamados, respectivamente, por: labels, redirects e disambiguations \ref{DBpedia-files-webpage}). E é possível obter o dump do conteúdo todas as páginas em uma língua na Wikipédia \ref{extraction-framework}. A partir destes o DBpedia Spotlight consegue extrair mapeamentos de candidatos segundo as quatro estratégias descritas a seguir respectivamente. %Como exemplo as Tabelas \ref{tab-berlim-labels}, \ref{tab-berlim-redirects} e \ref{tab-berlim-disambiguations} apresentam as linhas dos arquivos de títulos, redirecionamentos e desambiguações referentes a entidade ``Berlim'' na DBpedia em português. Por sua vez o dump não será apresentado neste trabalho pois sua visualização se torna mais fácil consultando a página referente na Wikipédia (http://pt.wikipedia.org/wiki/Berlim).
Na quais o recurso ``Berlin'' da DBpedia em inglês (http://dbpedia.org/resource/Berlin) será utilizada como exemplo.

% \begin{table}[h] \label{tab-berlim-labels}
% \caption{Conteúdo do arquivo labels para entidade ``Berlim``}
%  \begin{tabular}{l}
%  
%  \end{tabular}
% \end{table}
%  
% \begin{table}[h] \label{tab-berlim-redirects}
% \caption{Conteúdo do arquivo redirects para entidade ``Berlim``}
%  \begin{tabular}{l}
% <http://pt.dbpedia.org/resource/Estado\_de\_Berlim> \\<http://dbpedia.org/ontology/wikiPageRedirects> \\ <http://pt.dbpedia.org/resource/Berlim> .\\ \\
% <http://pt.dbpedia.org/resource/Área\_Metropolitana\_de\_Berlim> \\<http://dbpedia.org/ontology/wikiPageRedirects>\\ <http://pt.dbpedia.org/resource/Berlim> .\\ \\
% <http://pt.dbpedia.org/resource/Área\_metropolitana\_de\_berlim> \\<http://dbpedia.org/ontology/wikiPageRedirects>\\ <http://pt.dbpedia.org/resource/Berlim> .\\ \\
% <http://pt.dbpedia.org/resource/Estado\_de\_berlim> \\<http://dbpedia.org/ontology/wikiPageRedirects>\\ <http://pt.dbpedia.org/resource/Berlim> .\\ \\
% <http://pt.dbpedia.org/resource/Berlin> \\<http://dbpedia.org/ontology/wikiPageRedirects> \\<http://pt.dbpedia.org/resource/Berlim> .
%  \end{tabular}
% \end{table}
% 
% \begin{table}[h] \label{tab-berlim-disambiguations}
% \caption{Conteúdo do arquivo disambiguations para entidade ``Berlim``}
%  \begin{tabular}{l}
%  
%  \end{tabular}
% \end{table}

\subsection{Extração a partir dos títulos} \label{extracao-titulos}
\indent\indent Toda a página na DBpedia que descreve um recurso é identificada por uma URL composta do endereço da DBpedia para o determinado idioma de um identificador de que se trata de um recurso, idêntico para todas as identidades, e, por último, de um rótulo (\textit{label}) exclusivo da entidade, o qual também é chamado de título. Do exemplo teríamos:
\begin{itemize}
 \item Parte comum: ''http://dbpedia.org/page/''
 \item Rótulo/Título: ``Berlin''
\end{itemize}

A estratégia de extração de candidatos a partir dos títulos consiste em considerar o rótulo de todos os recursos como uma \textit{surface forms} da qual o respectivo recurso é candidato. Portanto, um candidato para a \textit{surface form}, ``berlin''\\footnote{Neste relatório cosideramos as variações de maíuscula e minúsculas, como sendo a mesma \textit{surface form}. Ou seja, ``berlin'' é considerado equivalente a: ``Berlin'', ``BERLIN'', etc.} é o resurso ``http://dbpedia.org/resource/Berlin''. Por esta estratégia tem-se, a natural relação de que todo o recurso da DBpedia no idioma será candidato de pelo menos a \textit{surface form} que o entitula.

\subsection{Extração a partir dos redirecionamentos} \label{extracao-redirecionamentos}
\indent\indent Uma recurso na DBpedia e na Wikipédia pode ser alcançado não apenas por seu título, podendo, diversos, serem acessados por outras formas de referenciamento. Sendo estes identificadores secundário chamados de redirecionamentos. Por exemplo, o recurso ``http://dbpedia.org/resource/Berlin'' também pode ser acessada pelos redirecionamentos relacionado pelo elemento ``DBpedia-owl:wikiPageRedirects'' da ontologia da DBpedia (descritos em http://dbpedia.org/resource/Berlin). A lista abaixo apresenta alguns destes:
\begin{itemize}
 \item Berlin,\_Germany
 \item City\_of\_Berlin
 \item Capital\_of\_East\_Germany
 \item Berlin-Zentrum
 \item Berlin.de
 \item Berlin\_(Germany)
 \item Federal\_State\_of\_Berlin
 \item Historical\_sites\_in\_berlin
 \item Land\_Berlin
\end{itemize}

É importante ressaltar que em qualquer uma das estratégia de extração de candidato o DBpedia Spotlight desconsidera durante os caracteres especiais adicionados pela DBpedia, afim de obter as \textit{surface forms} no formato original em que aparecem em textos. Obtendo, assim, a lista a seguir:
\begin{itemize}
 \item ''berlin, germany``
 \item ``city of berlin''
 \item ''capital of east germany``
 \item ``berlin-zentrum''
 \item ''berlin.de``
 \item ``berlin (germany)''
 \item ''federal state of berlin``
 \item ``historical sites in berlin''
 \item ''land berlin``
\end{itemize}

Observa-se que da mesma forma que uma ocorrência da palavra ''berlin`` em um texto tem chanses de se referir ao recurso ''http://dbpedia.org/resource/Berlin``, isto é válido para ''city of berlin``, ''berlin, germany`` e as demais obtidas do redirecionamento. Por isto, a estratégia de extração de candidatos a partir dos redirecionamentos, considerará as lista acima como \textit{surface forms} para as quais o recurso ''http://dbpedia.org/resource/Berlin`` é um candidato.

\subsection{Extração a partir das desambiguações} \label{extracao-disambiguacoes}
\indent\indent Uma expressão pode ter diversos significados, estas ambiguidades são relacionadas nas páginas de desambiguação da DBpedia (e Wikipédia). O DBpedia Spotlight é capaz de extrair utilizando-se de um processo inverso ao realizado pelo estratégia a partir dos redirecionamentos.

Seja a página de desambiguação da DBpedia: ''http://dbpedia.org/page/Berlin\_(disambiguation)``. Removendo as estruturas utilizadas pela DBpedia para construção da URL e identificação de que se trata de uma página de desambiguação, tem-se a \textit{surface form}: ''berlin``\\footnote{E realizando as mesmas formatações descritas na Seção \ref{extracao-titulos}.

As páginas de desambigução apresentam no elemento ''dbpedia-owl:wikiPageDisambiguates`` da ontologia todos os recursos relacionados com a \textit{surface form} obtida do título da página de desambiguação. Alguns destes recursos para a página de desambiguação de exemplo estão listados abaixo:
\begin{itemize}
 \item http://dbpedia.org/resource/Berlin
 \item http://dbpedia.org/resource/Berlin,\_New\_Hampshire
 \item http://dbpedia.org/resource/Berlin,\_Wisconsin
 \item http://dbpedia.org/resource/Berlin,\_Marathon\_County,\_Wisconsin
 \item http://dbpedia.org/resource/Berlin\_(Amtrak\_station)
 \item http://dbpedia.org/resource/Berlin\_(band)
 \item http://dbpedia.org/resource/Mount\_Berlin
 \item http://dbpedia.org/resource/SS\_Berlin
 \item http://dbpedia.org/resource/Berlin\_(comic)
 \item http://dbpedia.org/resource/Berlin\_(surname)
 \item http://dbpedia.org/resource/East\_Berlin\_(disambiguation)
 \item http://dbpedia.org/resource/West\_Berlin\_(disambiguation)
\end{itemize}

Observa-se que além do recurso utilizado como exemplo para as estratégias anteriores (Seções \ref{extracao-titulos} e \ref{extracao-redirecionamentos}), o qual é a entidade referente a capital da Alemanha, a expressão ''berlin`` também está associada a outras cidades, tal qual a entidades de outras natureza como: estação de trem, banda de música, monte, embarcação, série de história em quadrinhos, sobrenome.

Portanto, está estratégia de extração de mapa de candidatos irá considerar todos estes recursos como candidatos para a \textit{surface form} extraída do título da página de desambiguação, no caso ''berlin``. Esta, ainda, terá como candidatos os recursos relacionados as páginas de desambiguação que estiverem listadas no elemento ''dbpedia-owl:wikiPageDisambiguates``. Para o exemplo, todos os recursos obtidos pela estratégia de desambiguação para as páginas ''http://dbpedia.org/resource/East\_Berlin\_(disambiguation)`` e ''http://dbpedia.org/resource/West\_Berlin\_(disambiguation)`` serão candidatos para ''berlin``, tal qual para a \textit{surface form} obtida do título de suas respectivas páginas.

\subsection{Extração a partir das ocorrências} \label{extracao-ocorrencias}
\indent\indent Diferentemente das estratégias anteriores, a extração a partir das ocorrências no contúdo de uma página da Wikipédia não utiliza-se da estrutura do nome dos recursos ou dos elementos da ontologia definida na DBpedia. Por sua vez, é realizado processando o conteúdo cada página da Wikipédia (as quais são obtidas através de dump desta para um idioma realizado pelo software livre \ref{extraction-framework}).

Este processamento é feito de forma a identificar as expressões de mais relevantes no texto contido na página. As quais serão consideradas \textit{surface forms} relacionadas ao recurso da página. Por exemplo, seja a página da Wikipédia: ''http://en.wikipedia.org/wiki/Berlin'' e algumas expressões relevantes identificadas listadas a seguir:
\begin{itemize}
 \item ``berlin''
 \item ``germany's largest city''
 \item ``capital city of germany''
 \item ``capital of the kingdom of prussia''
 \item ``berliner''
\end{itemize}

Sabe-se que que o recurso da DBpedia referente a está página é o recurso: ``http://dbpedia.org/resource/Berlin''. O qual será mapeado como candidato para todas as \textit{surface forms} listadas acima.


\section{Experimentos} \label{experimentos}
\indent\indent Inicialmente foram construídos os mapas de candidatos por cada uma das estratégias descritas na Seção \ref{extracao-mapa-candidatos} separadamente. Obtendo, assim, os mapas a partir: dos títulos (\textit{T}), dos redirecionamentos (\textit{R}), das desambiguações (\textit{D}) e das ocorrências (\textit{O}). Isto foi realizado utilizando o Algoritmo \ref{alg-extractCandidateMap} tendo como entrada o \textit{dump} de todos os arquivos da DBpedia \ref{DBpediaOrg} em inglês.

Para a avaliação da qualidade destes quatro mapas, foram utilizados três cospora obtidos na literatura com textos em inglês (Seção \ref{corpora}). Onde, para cada entidade destes foram verificadas se o recurso ao qual estavam anotados era um dos candidatos relacionados a entidade pelo mapa em questão (Algoritmo \ref{alg-extractCandidateMap}). Então, foram calculadas as seguintes métricas (conforme proposto por \cite{citet-Hachey2012}):

\begin{itemize}
 \item \textit{Acurácia}: é ... \todo{descrever}
 \item \todo{descrição métricas utilizadas segundo Hachey}
\end{itemize}

\subsection{Corpora} \label{corpora}
\indent\indent Os corpora utilizados foram: Mille \& Witten Corpus (\cite{citep-MW-vide-apresentacao-spotters}), CSAW Corpus (\cite{citep-CSAW-vide-apresentacao-spotters}) e Aida-CoNLL-Yago2 Corpus (\cite{citep-MW-vide-apresentacao-spotters}). Todos estes compostos A Tabela \ref{tab-corpora} apresenta suas principais características:

\begin{table} \label{tab-corpora}
\caption{Características dos corpora utilizados}
\centering
    \begin{tabular}{|c|c|c|c|}
    \hline
    ~                                    & {\bf M\&W} & {\bf CSAW}               & ~ {\bf Aida CoNLL} ~ \\ \hline
    {\bf Número de Artigos}   & 50                    & 103                                 & 1393                        \\ \hline
    ~ {\bf Número de Entidades} ~ & 706                   & 12099                               & 34929                       \\ \hline
    {\bf Estilo dos Artigos}  & ~ Jornalístico ~          & ~ Jornalístico ~                & ~ Jornalístico ~                \\
    ~                                    & ~                     & ~ e {\it Wikipédia} ~ & ~                           \\ \hline
    \end{tabular}
\end{table}

\subsection{Resultados} \label{resultados}
\indent\indent A
\todo{Dados Carolina}

\section{Conclusão} \label{conclusao}
\indent\indent A
\todo{Depende dos resultados}

\bibliographystyle{splncs03}
\bibliography{eswc}

\end{document}
