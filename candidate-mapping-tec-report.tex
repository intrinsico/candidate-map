%\documentclass[10pt,a4paper]{article}
\documentclass[10pt,a4paper]{llncs}

\usepackage{makeidx}  % allows for indexgeneration
\usepackage[utf8]{inputenc}
\usepackage[cmex10]{amsmath}
\usepackage{amsfonts}
\usepackage{hyperref}
\usepackage{color}
\usepackage{cleveref}
\usepackage{graphicx}
\usepackage{caption}
\usepackage{subcaption}

\newcommand{\todo}[1]{{\color{red}\textsf{\textbf{TODO}}: #1}}
\newcommand{\uri}[1]{\texttt{#1}}
\newcommand{\word}[1]{`#1'}
\newcommand{\model}[1]{\textsf{#1}}
\crefname{table}{Table}{Tables}

\title{Mapeamento de Candidatos}
\author{Alexandre Cançado Cardoso}
\institute{Intrinsic Soluções em Informática M.E.}
\date{\today}

\begin{document}
\maketitle

\begin{abstract}
\end{abstract}

\section{Introdução} \label{introducao}
\indent\indent Em um proceso de anotação semântica a definição dos possiveis recursos candidatos para cada nome de entidade é uma tarefa fundamental. Pois é a partir destas relação que as anotações paras as entidades interessantes serão escolhidas. O conjunto das relações nome de entidades, recursos é denominado de mapa de candidatos. E pode ser definido de diversas formas.

Podendo o uso das diferentes formas de obtenção dos candidatos gerar mapas de qualidades distintas. Neste contexto, a avaliação de quais são as melhores estratégias de extração de candidatos se torna importante. Nas seções a seguir, quatro destas técnicas seram comparadas.


\section{Estratégias de Extração de Mapa de Candidatos} \label{extracao-mapa-candidatos}
\indent\indent As estratégias de extração de mapa de candidatos abordadas por este trabalho são a partir dos titulos, dos redirecionamentos, das disambiguações e das ocorrências no conteúdo das páginas.

\subsection{Extração a partir dos títulos} \label{extracao-titulos}
\indent\indent A
\todo{seção}

\subsection{Extração a partir dos redirecionamentos} \label{extracao-redirecionamentos}
\indent\indent A
\todo{seção}

\subsection{Extração a partir das disambiguações} \label{extracao-disambiguacoes}
\indent\indent A
\todo{seção}

\subsection{Extração a partir das ocorrências} \label{extracao-ocorrencias}
\indent\indent A
\todo{seção}


\section{Experimentos} \label{experimentos}
\indent\indent Inicialmente foram construidos os mapas de candidatos por cada uma das estratégias descritas na Seção \ref{extracao-mapa-candidatos} separadamente. Obtendo, assim, os mapas a partir: dos titulos (\textit{T}), dos redirecionamentos (\textit{R}), das disambiguações (\textit{D}) e das ocorrências (\textit{O}). Isto foi realizado utilizando o Algoritmo \ref{alg-extractCandidateMap} tendo como entrada o \textit{dump} de todos os arquivos da DBpedia \ref{DBpediaOrg} em inglês.

Para a avaliação da qualidade destes quatro mapas, foram utilizados três cospora obtidos na literatura com textos em inglês (Seção \ref{corpora}). Onde, para cada entidade destes foram verificadas se o recurso ao qual estavam anotados era um dos candidatos relacionados a entidade pelo mapa em questão (Algoritmo \ref{alg-extractCandidateMap}). Então, foram calculadas as seguintes métricas (conforme proposto por \cite{citet-Hachey2012}):

\begin{itemize}
 \item \textit{Acurácia}: é ... \todo{descrever}
 \item \todo{descrição métricas utilizadas segundo Hachey}
\end{itemize}

\subsection{Corpora} \label{corpora}
\indent\indent Os corpora utilizados foram: Mille \& Witten Corpus (\cite{citep-MW-vide-apresentacao-spotters}), CSAW Corpus (\cite{citep-CSAW-vide-apresentacao-spotters}) e Aida-CoNLL-Yago2 Corpus (\cite{citep-MW-vide-apresentacao-spotters}). Todos estes compostos A Tabela \ref{tab-corpora} apresenta suas principais características:

\begin{table} \label{tab-corpora}
\caption{Características dos corpora utilizados}
\centering
    \begin{tabular}{|c|c|c|c|}
    \hline
    ~                                    & {\bf M\&W} & {\bf CSAW}               & ~ {\bf Aida CoNLL} ~ \\ \hline
    {\bf Número de Artigos}   & 50                    & 103                                 & 1393                        \\ \hline
    ~ {\bf Número de Entidades} ~ & 706                   & 12099                               & 34929                       \\ \hline
    {\bf Estilo dos Artigos}  & ~ Jornalístico ~          & ~ Jornalístico ~                & ~ Jornalístico ~                \\
    ~                                    & ~                     & ~ e {\it Wikipedia} ~ & ~                           \\ \hline
    \end{tabular}
\end{table}

\subsection{Resultados} \label{resultados}
\indent\indent A
\todo{Dados Carolina}

\section{Conclusão} \label{conclusao}
\indent\indent A
\todo{Depende dos resultados}

\bibliographystyle{splncs03}
\bibliography{eswc}

\end{document}
